%%%%%%%%%%%%%%%%%%%%%%%%%%%%%%%%%%%%%%%%%
% Lachaise Assignment
% LaTeX Template
% Version 1.0 (26/6/2018)
%
% This template originates from:
% http://www.LaTeXTemplates.com
%
% Authors:
% Marion Lachaise & François Févotte
% Vel (vel@LaTeXTemplates.com)
%
% License:
% CC BY-NC-SA 3.0 (http://creativecommons.org/licenses/by-nc-sa/3.0/)
% 
%%%%%%%%%%%%%%%%%%%%%%%%%%%%%%%%%%%%%%%%%

%----------------------------------------------------------------------------------------
%	PACKAGES AND OTHER DOCUMENT CONFIGURATIONS
%----------------------------------------------------------------------------------------

\documentclass[main]{subfiles}

%%%%%%%%%%%%%%%%%%%%%%%%%%%%%%%%%%%%%%%%%
% Lachaise Assignment
% Structure Specification File
% Version 1.0 (26/6/2018)
%
% This template originates from:
% http://www.LaTeXTemplates.com
%
% Authors:
% Marion Lachaise & François Févotte
% Vel (vel@LaTeXTemplates.com)
%
% License:
% CC BY-NC-SA 3.0 (http://creativecommons.org/licenses/by-nc-sa/3.0/)
% 
%%%%%%%%%%%%%%%%%%%%%%%%%%%%%%%%%%%%%%%%%

%----------------------------------------------------------------------------------------
%	PACKAGES AND OTHER DOCUMENT CONFIGURATIONS
%----------------------------------------------------------------------------------------

\usepackage{amsmath,amsfonts,stmaryrd,amssymb} % Math packages

\usepackage{enumerate} % Custom item numbers for enumerations

\usepackage[ruled]{algorithm2e} % Algorithms

\usepackage[framemethod=tikz]{mdframed} % Allows defining custom boxed/framed environments

\usepackage{listings} % File listings, with syntax highlighting
\lstset{
	basicstyle=\ttfamily, % Typeset listings in monospace font
}

%----------------------------------------------------------------------------------------
%	DOCUMENT MARGINS
%----------------------------------------------------------------------------------------

\usepackage{geometry} % Required for adjusting page dimensions and margins

\geometry{
	paper=a4paper, % Paper size, change to letterpaper for US letter size
	top=2.5cm, % Top margin
	bottom=3cm, % Bottom margin
	left=2.5cm, % Left margin
	right=2.5cm, % Right margin
	headheight=14pt, % Header height
	footskip=1.5cm, % Space from the bottom margin to the baseline of the footer
	headsep=1.2cm, % Space from the top margin to the baseline of the header
	%showframe, % Uncomment to show how the type block is set on the page
}

%----------------------------------------------------------------------------------------
%	FONTS
%----------------------------------------------------------------------------------------

\usepackage[utf8]{inputenc} % Required for inputting international characters
\usepackage[T1]{fontenc} % Output font encoding for international characters

\usepackage{XCharter} % Use the XCharter fonts

%----------------------------------------------------------------------------------------
%	COMMAND LINE ENVIRONMENT
%----------------------------------------------------------------------------------------

% Usage:
% \begin{commandline}
%	\begin{verbatim}
%		$ ls
%		
%		Applications	Desktop	...
%	\end{verbatim}
% \end{commandline}

\mdfdefinestyle{commandline}{
	leftmargin=10pt,
	rightmargin=10pt,
	innerleftmargin=15pt,
	middlelinecolor=black!50!white,
	middlelinewidth=2pt,
	frametitlerule=false,
	backgroundcolor=black!5!white,
	frametitle={Command Line},
	frametitlefont={\normalfont\sffamily\color{white}\hspace{-1em}},
	frametitlebackgroundcolor=black!50!white,
	nobreak,
}

% Define a custom environment for command-line snapshots
\newenvironment{commandline}{
	\medskip
	\begin{mdframed}[style=commandline]
}{
	\end{mdframed}
	\medskip
}

%----------------------------------------------------------------------------------------
%	FILE CONTENTS ENVIRONMENT
%----------------------------------------------------------------------------------------

% Usage:
% \begin{file}[optional filename, defaults to "File"]
%	File contents, for example, with a listings environment
% \end{file}

\mdfdefinestyle{file}{
	innertopmargin=1.6\baselineskip,
	innerbottommargin=0.8\baselineskip,
	topline=false, bottomline=false,
	leftline=false, rightline=false,
	leftmargin=2cm,
	rightmargin=2cm,
	singleextra={%
		\draw[fill=black!10!white](P)++(0,-1.2em)rectangle(P-|O);
		\node[anchor=north west]
		at(P-|O){\ttfamily\mdfilename};
		%
		\def\l{3em}
		\draw(O-|P)++(-\l,0)--++(\l,\l)--(P)--(P-|O)--(O)--cycle;
		\draw(O-|P)++(-\l,0)--++(0,\l)--++(\l,0);
	},
	nobreak,
}

% Define a custom environment for file contents
\newenvironment{file}[1][File]{ % Set the default filename to "File"
	\medskip
	\newcommand{\mdfilename}{#1}
	\begin{mdframed}[style=file]
}{
	\end{mdframed}
	\medskip
}

%----------------------------------------------------------------------------------------
%	NUMBERED QUESTIONS ENVIRONMENT
%----------------------------------------------------------------------------------------

% Usage:
% \begin{question}[optional title]
%	Question contents
% \end{question}

\mdfdefinestyle{question}{
	innertopmargin=1.2\baselineskip,
	innerbottommargin=0.8\baselineskip,
	roundcorner=5pt,
	nobreak,
	singleextra={%
		\draw(P-|O)node[xshift=1em,anchor=west,fill=white,draw,rounded corners=5pt]{%
		Question \theQuestion\questionTitle};
	},
}

\newcounter{Question} % Stores the current question number that gets iterated with each new question

% Define a custom environment for numbered questions
\newenvironment{question}[1][\unskip]{
	\bigskip
	\stepcounter{Question}
	\newcommand{\questionTitle}{~#1}
	\begin{mdframed}[style=question]
}{
	\end{mdframed}
	\medskip
}

%----------------------------------------------------------------------------------------
%	WARNING TEXT ENVIRONMENT
%----------------------------------------------------------------------------------------

% Usage:
% \begin{warn}[optional title, defaults to "Warning:"]
%	Contents
% \end{warn}

\mdfdefinestyle{warning}{
	topline=false, bottomline=false,
	leftline=false, rightline=false,
	nobreak,
	singleextra={%
		\draw(P-|O)++(-0.5em,0)node(tmp1){};
		\draw(P-|O)++(0.5em,0)node(tmp2){};
		\fill[black,rotate around={45:(P-|O)}](tmp1)rectangle(tmp2);
		\node at(P-|O){\color{white}\scriptsize\bf !};
		\draw[very thick](P-|O)++(0,-1em)--(O);%--(O-|P);
	}
}

% Define a custom environment for warning text
\newenvironment{warn}[1][Warning:]{ % Set the default warning to "Warning:"
	\medskip
	\begin{mdframed}[style=warning]
		\noindent{\textbf{#1}}
}{
	\end{mdframed}
}

%----------------------------------------------------------------------------------------
%	INFORMATION ENVIRONMENT
%----------------------------------------------------------------------------------------

% Usage:
% \begin{info}[optional title, defaults to "Info:"]
% 	contents
% 	\end{info}

\mdfdefinestyle{info}{%
	topline=false, bottomline=false,
	leftline=false, rightline=false,
	nobreak,
	singleextra={%
		\fill[black](P-|O)circle[radius=0.4em];
		\node at(P-|O){\color{white}\scriptsize\bf i};
		\draw[very thick](P-|O)++(0,-0.8em)--(O);%--(O-|P);
	}
}

% Define a custom environment for information
\newenvironment{info}[1][Info:]{ % Set the default title to "Info:"
	\medskip
	\begin{mdframed}[style=info]
		\noindent{\textbf{#1}}
}{
	\end{mdframed}
}
 % Include the file specifying the document structure and custom commands

%----------------------------------------------------------------------------------------
%	ASSIGNMENT INFORMATION
%----------------------------------------------------------------------------------------

\title{Statistic Learning Protfolio: Linear Regression Analysis} % Title of the assignment

\author{Yu Yangcheng\\ \texttt{yuyc23@mails.tsinghua.edu.cn}} % Author name and email address

\date{Tsinghua University--- \today} % University, school and/or department name(s) and a date

%----------------------------------------------------------------------------------------

\begin{document}

\maketitle % Print the title

%----------------------------------------------------------------------------------------
%	SECTIOHN 2
%----------------------------------------------------------------------------------------

\section{Simple Linear Regression} % Numbered section

\subsection{Building the structure}
The simple linear regression(SLE) model's formula is:
\begin{equation}
    Y_i = \beta_0 + \beta_1X_i + \varepsilon_i
\end{equation}
Where $Y_i$ is the response variable, $X_i$ is the predictor variable, $\beta_0$ is the intercept, $\beta_1$ is the slope, and $\varepsilon_i$ is the error term. The goal of simple linear regression is to estimate the coefficients $\beta_0$ and $\beta_1$ such that the error term $\varepsilon_i$ is minimized. The error term $\varepsilon_i$ is the difference between the observed value of the response variable and the value predicted by the model. The least squares method is used to estimate the coefficients. The least squares estimates(LSE) minimizes the sum of the squared errors. The sum of the squared errors is the sum of the squared differences between each observed value and the predicted value.
Simple linear regression is built under 4 assumptions(LINE):
\begin{enumerate}
    \item Linear Function: The mean of the response($E(Y_i)$), at each value of the predictor($X_i$), is a linear function of $X_i$.
    \item Independent: The errors($\varepsilon_i$) are independent.
    \item Normally Distributed: The errors($\varepsilon_i$) are normally distributed.
    \item Equal Variance: The errors($\varepsilon_i$) have Equal variances.
\end{enumerate}
Why should we have these assumptions? Firstly, Taylor expansion tells us that most of the functions can be approximated by a linear function, as long as the range is small enough. Secondly, independent errors make writting the joint distribution of the errors easier(We just need to do production). Thirdly, if we assume the errors are normally distributed, we can use the "$3 \sigma$" principle to calculate the confidence interval. Finally, equal variance can reduce the complexity of the model.
\begin{info}
    Review what we just have learnt: SLR's LINE assumptions({\bf L}inear, {\bf I}ndepent, {\bf N}ormally distributed, {\bf E}qual variance). In the following sections, we will take them as granted.
\end{info}
\subsection{Estimate the coefficients}
In SLR, we have {\bf 3} coefficients to estimate: $\beta_0$, $\beta_1$, and $\varepsilon_i$. The last one, $\varepsilon_i$, helps us to evaluate the model's performance. (Obviously, the smaller $\varepsilon_i$ is, the better the model is.) \\
To estimate them, we will have 2 methods: {\bf 1.} least squares estimates(LSE), and {\bf 2.} Maximum Likelihood Estimates(MLE). The LSE method minimizes the sum of the squared errors, while the MLE method maximizes the likelihood function. 
\subsubsection{Least Squares Estimates}
In the LSE method, we want to choose $\beta_0$ and $\beta_1$ to minimize
$$ Q = \sum(Y_i-\beta_0-\beta_1 X_i)^{2} $$
Let's do calculus:
\begin{align*}
    \frac{\partial Q}{\partial \beta_0} &= -2\sum(Y_i-\beta_0-\beta_1 X_i) = 0 \\
    \frac{\partial Q}{\partial \beta_1} &= -2\sum(Y_i-\beta_0-\beta_1 X_i)X_i = 0
\end{align*}
Solve the equations, we get:
\begin{align*}
    \hat{\beta_1} &= \frac{\sum(X_i-\bar{X})(Y_i-\bar{Y})}{\sum(X_i-\bar{X})^2} \\
    \hat{\beta_0} &= \bar{Y} - \hat{\beta_1}\bar{X}
\end{align*}
Replace $\hat{\beta_0}$ and $\hat{\beta_1}$ with $b_0$, $b_1$, we find:
$$ \sum_{i=1}^{n}e_i = 0, \quad  \sum_{i=1}^{n}X_i e_i = 0$$
\begin{warn}
    The degress of freedom (df) of $e_1,e_2\dots e_n$ is $n-2$, because every time we add a constraint, we reduce the df by 1. Therefore, 
    $$ s ^{2} = \frac{\sum_{i=1}^{n}e_i^2}{n-1}=\frac{SSE}{df_E}=MSE $$
    SSE is the short of Sum of Squared Errors, and MSE is the short of Mean Squared Errors.
\end{warn}
\subsubsection{Maximum Likelihood Estimates}
In the MLE method, we want to maximize the likelihood function:
$$ L(\beta_0,\beta_1,\sigma ^{2} | X_i,Y_i) = \prod_{i=1}^{n}f(Y_i|X_i;\beta_0,\beta_1,\sigma ^{2}) $$
Where $f(Y_i|X_i;\beta_0,\beta_1,\sigma ^{2})$ is the probability density function of $Y_i$ given $X_i$. We assume that $Y_i$ is normally distributed, so we have:
$$ f(Y_i|X_i;\beta_0,\beta_1,\sigma ^{2}) = \frac{1}{\sqrt{2\pi}\sigma}exp\left(-\frac{(Y_i-\beta_0-\beta_1X_i)^2}{2\sigma^2}\right) $$
To get the MLE, we need to find $argmax_{\beta_0, \beta_1, \sigma ^{2}} L(\beta_0, \beta_1, \sigma ^{2})$, that's to say, $argmax_{\beta_0, \beta_1, \sigma ^{2}}-\log L(\beta_0,\beta_1,\sigma^2)$
Use calculus, we get:
\begin{align*}
    &\hat{\beta}_0^{ml}=b_0, \quad \hat{\beta}_1^{ml}=b_1, \\
    &\hat{\sigma}^{2^{ml}}=\frac{\sum_{i=1}^{n}e_i^2}{n}
\end{align*}
Why do we get different MSE(mean squared errors) in LSE and MLE? Because in MLE, we assume the normal distribution of $Y_i$ at the beginning, which allow us to get n df. This also results in: ML estimates rely on distributional assumptions, while LS estimates do not.
\end{document}
