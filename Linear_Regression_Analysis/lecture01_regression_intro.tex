%%%%%%%%%%%%%%%%%%%%%%%%%%%%%%%%%%%%%%%%%
% Lachaise Assignment
% LaTeX Template
% Version 1.0 (26/6/2018)
%
% This template originates from:
% http://www.LaTeXTemplates.com
%
% Authors:
% Marion Lachaise & François Févotte
% Vel (vel@LaTeXTemplates.com)
%
% License:
% CC BY-NC-SA 3.0 (http://creativecommons.org/licenses/by-nc-sa/3.0/)
% 
%%%%%%%%%%%%%%%%%%%%%%%%%%%%%%%%%%%%%%%%%

%----------------------------------------------------------------------------------------
%	PACKAGES AND OTHER DOCUMENT CONFIGURATIONS
%----------------------------------------------------------------------------------------

\documentclass[main]{subfiles}

%%%%%%%%%%%%%%%%%%%%%%%%%%%%%%%%%%%%%%%%%
% Lachaise Assignment
% Structure Specification File
% Version 1.0 (26/6/2018)
%
% This template originates from:
% http://www.LaTeXTemplates.com
%
% Authors:
% Marion Lachaise & François Févotte
% Vel (vel@LaTeXTemplates.com)
%
% License:
% CC BY-NC-SA 3.0 (http://creativecommons.org/licenses/by-nc-sa/3.0/)
% 
%%%%%%%%%%%%%%%%%%%%%%%%%%%%%%%%%%%%%%%%%

%----------------------------------------------------------------------------------------
%	PACKAGES AND OTHER DOCUMENT CONFIGURATIONS
%----------------------------------------------------------------------------------------

\usepackage{amsmath,amsfonts,stmaryrd,amssymb} % Math packages

\usepackage{enumerate} % Custom item numbers for enumerations

\usepackage[ruled]{algorithm2e} % Algorithms

\usepackage[framemethod=tikz]{mdframed} % Allows defining custom boxed/framed environments

\usepackage{listings} % File listings, with syntax highlighting
\lstset{
	basicstyle=\ttfamily, % Typeset listings in monospace font
}

%----------------------------------------------------------------------------------------
%	DOCUMENT MARGINS
%----------------------------------------------------------------------------------------

\usepackage{geometry} % Required for adjusting page dimensions and margins

\geometry{
	paper=a4paper, % Paper size, change to letterpaper for US letter size
	top=2.5cm, % Top margin
	bottom=3cm, % Bottom margin
	left=2.5cm, % Left margin
	right=2.5cm, % Right margin
	headheight=14pt, % Header height
	footskip=1.5cm, % Space from the bottom margin to the baseline of the footer
	headsep=1.2cm, % Space from the top margin to the baseline of the header
	%showframe, % Uncomment to show how the type block is set on the page
}

%----------------------------------------------------------------------------------------
%	FONTS
%----------------------------------------------------------------------------------------

\usepackage[utf8]{inputenc} % Required for inputting international characters
\usepackage[T1]{fontenc} % Output font encoding for international characters

\usepackage{XCharter} % Use the XCharter fonts

%----------------------------------------------------------------------------------------
%	COMMAND LINE ENVIRONMENT
%----------------------------------------------------------------------------------------

% Usage:
% \begin{commandline}
%	\begin{verbatim}
%		$ ls
%		
%		Applications	Desktop	...
%	\end{verbatim}
% \end{commandline}

\mdfdefinestyle{commandline}{
	leftmargin=10pt,
	rightmargin=10pt,
	innerleftmargin=15pt,
	middlelinecolor=black!50!white,
	middlelinewidth=2pt,
	frametitlerule=false,
	backgroundcolor=black!5!white,
	frametitle={Command Line},
	frametitlefont={\normalfont\sffamily\color{white}\hspace{-1em}},
	frametitlebackgroundcolor=black!50!white,
	nobreak,
}

% Define a custom environment for command-line snapshots
\newenvironment{commandline}{
	\medskip
	\begin{mdframed}[style=commandline]
}{
	\end{mdframed}
	\medskip
}

%----------------------------------------------------------------------------------------
%	FILE CONTENTS ENVIRONMENT
%----------------------------------------------------------------------------------------

% Usage:
% \begin{file}[optional filename, defaults to "File"]
%	File contents, for example, with a listings environment
% \end{file}

\mdfdefinestyle{file}{
	innertopmargin=1.6\baselineskip,
	innerbottommargin=0.8\baselineskip,
	topline=false, bottomline=false,
	leftline=false, rightline=false,
	leftmargin=2cm,
	rightmargin=2cm,
	singleextra={%
		\draw[fill=black!10!white](P)++(0,-1.2em)rectangle(P-|O);
		\node[anchor=north west]
		at(P-|O){\ttfamily\mdfilename};
		%
		\def\l{3em}
		\draw(O-|P)++(-\l,0)--++(\l,\l)--(P)--(P-|O)--(O)--cycle;
		\draw(O-|P)++(-\l,0)--++(0,\l)--++(\l,0);
	},
	nobreak,
}

% Define a custom environment for file contents
\newenvironment{file}[1][File]{ % Set the default filename to "File"
	\medskip
	\newcommand{\mdfilename}{#1}
	\begin{mdframed}[style=file]
}{
	\end{mdframed}
	\medskip
}

%----------------------------------------------------------------------------------------
%	NUMBERED QUESTIONS ENVIRONMENT
%----------------------------------------------------------------------------------------

% Usage:
% \begin{question}[optional title]
%	Question contents
% \end{question}

\mdfdefinestyle{question}{
	innertopmargin=1.2\baselineskip,
	innerbottommargin=0.8\baselineskip,
	roundcorner=5pt,
	nobreak,
	singleextra={%
		\draw(P-|O)node[xshift=1em,anchor=west,fill=white,draw,rounded corners=5pt]{%
		Question \theQuestion\questionTitle};
	},
}

\newcounter{Question} % Stores the current question number that gets iterated with each new question

% Define a custom environment for numbered questions
\newenvironment{question}[1][\unskip]{
	\bigskip
	\stepcounter{Question}
	\newcommand{\questionTitle}{~#1}
	\begin{mdframed}[style=question]
}{
	\end{mdframed}
	\medskip
}

%----------------------------------------------------------------------------------------
%	WARNING TEXT ENVIRONMENT
%----------------------------------------------------------------------------------------

% Usage:
% \begin{warn}[optional title, defaults to "Warning:"]
%	Contents
% \end{warn}

\mdfdefinestyle{warning}{
	topline=false, bottomline=false,
	leftline=false, rightline=false,
	nobreak,
	singleextra={%
		\draw(P-|O)++(-0.5em,0)node(tmp1){};
		\draw(P-|O)++(0.5em,0)node(tmp2){};
		\fill[black,rotate around={45:(P-|O)}](tmp1)rectangle(tmp2);
		\node at(P-|O){\color{white}\scriptsize\bf !};
		\draw[very thick](P-|O)++(0,-1em)--(O);%--(O-|P);
	}
}

% Define a custom environment for warning text
\newenvironment{warn}[1][Warning:]{ % Set the default warning to "Warning:"
	\medskip
	\begin{mdframed}[style=warning]
		\noindent{\textbf{#1}}
}{
	\end{mdframed}
}

%----------------------------------------------------------------------------------------
%	INFORMATION ENVIRONMENT
%----------------------------------------------------------------------------------------

% Usage:
% \begin{info}[optional title, defaults to "Info:"]
% 	contents
% 	\end{info}

\mdfdefinestyle{info}{%
	topline=false, bottomline=false,
	leftline=false, rightline=false,
	nobreak,
	singleextra={%
		\fill[black](P-|O)circle[radius=0.4em];
		\node at(P-|O){\color{white}\scriptsize\bf i};
		\draw[very thick](P-|O)++(0,-0.8em)--(O);%--(O-|P);
	}
}

% Define a custom environment for information
\newenvironment{info}[1][Info:]{ % Set the default title to "Info:"
	\medskip
	\begin{mdframed}[style=info]
		\noindent{\textbf{#1}}
}{
	\end{mdframed}
}
 % Include the file specifying the document structure and custom commands

%----------------------------------------------------------------------------------------
%	ASSIGNMENT INFORMATION
%----------------------------------------------------------------------------------------

\title{Statistic Learning Protfolio: Linear Regression Analysis} % Title of the assignment

\author{Yu Yangcheng\\ \texttt{yuyc23@mails.tsinghua.edu.cn}} % Author name and email address

\date{Tsinghua University--- \today} % University, school and/or department name(s) and a date

%----------------------------------------------------------------------------------------

\begin{document}

\maketitle % Print the title

%----------------------------------------------------------------------------------------
%	INTRODUCTION
%----------------------------------------------------------------------------------------

\section*{Introduction} % Unnumbered section

This document is written for self-revision of the course Linear Regression Analysis and recognize my learning process. Therefore, it may not be a encyclopedic reference for this subject, but a personal record of my learning. Now, it's time for our journey to begin!

\begin{info} % Information block
  Considering my learning background, in these documents, I will focus more on the math explanation of the concepts, and try to make the concepts more intuitive. I will also try to connect the concepts with the real world, and make the concepts more vivid.
\end{info}

%----------------------------------------------------------------------------------------
%	SECTIOHN 1
%----------------------------------------------------------------------------------------

\section{History of Regression} % Numbered section

\subsection{Galton's Discovery}
Linear regression is a statistical method that models the linear relationship between two variables. The word "Regression", was originate from Galton, Darwin's cousin, who first used it to describe the phenomenon that the offspring of parents with extreme characteristics tend to have characteristics that are closer to the average. \par
Galton's discovery comes from Galton board, also called "quincunx" in PPT, which is a device consist of different layers of pins. When a ball is dropped from the top, it will bounce off the pins and finally fall into the bottom. The final distribution of the balls is a normal distribution.\par
By observing this ball's distribution between different layers, Galton found that the balls tend to fall into the middle, and the distribution tends to be more and more distracted in the falling process. This leads to a paradox: if the offspring of parents with extreme characteristics would inherit their parents' characteristics, how could species keep stable? All in all, a species is recognized by some steady characteristics. \par
Galton explain this by introducing the concept of regression. He found that the offspring of parents with extreme characteristics tend to have characteristics that are closer to the average. It doesn't mean that the inheritance is not exist, taller parents' children may still be more possible to be tall, but the bias will be smaller. For example, if the father is 2 meters tall(like Yao ming), the son may be 1.9 meters tall(Yao ming's daughter seems to be very tall, too). In long-term development, the regression effect will contain the accumulation of the variation. \par

%----------------------------------------------------------------------------------------
%	SECTION 2
%----------------------------------------------------------------------------------------

\subsection{Regression \& Correlation}

Regression and Correlation are two closely related concepts. When you collected a series of data, you may want to draw the regression line to describe the relationship between the variables, and calculate the slope of the regression line. The slope of the regression line is always between -1 and 1, which is called the correlation coefficient. 
The slope's value have 3 cases:
\begin{itemize}
    \item If the slope is 0, it means that x,y does't have linear relationship. \textbf{But} x,y could have non-linear relationship, eg. $y=x^2$ (prove it!).
    \item If the slop $\in (-1,1)$ \textbackslash $\{0\}$, it means that x,y could have linear relationship. The closer the slope is to 1, the stronger the linear relationship is. The sign of the slope indicates the direction of the relationship.
    \item If the slop is $\pm 1$, it means that x,y are completely linear related.
\end{itemize}

% Warning text, with a custom title
\begin{warn}[Notice:]
  No matter what data you choose, you will always get a correlation coefficient between -1 and 1. This is \textbf{a game of mathematics}, having nothing to do with genes, magic, or anything else. In Galton's research, he studied the relationship between brother's height, and found the regression phenomenon as well. As we all know, it's unfair to clarify that brothers could inherit each other's excellent genes... \par 
\end{warn}

%----------------------------------------------------------------------------------------
%	SECTION 3
%----------------------------------------------------------------------------------------

\subsection{Math Explanation of Regression}
If F is the n-1 th. generation's distribution, S is the n th. generation and $X_n$ is the variation in the n th. turn. We assume that the species is stable, meaning that $Var(S)=Var(F)$. Let $\rho (F,S)$ be the correlation coefficient of two variables $F$ and $S$. Then we have 
$$ \rho(F,S)=\frac{Cov(F,S)}{\sqrt{Var(F)Var(S)}} = \frac{Cov(F,F+X_n)}{Var(F)Var(S)}=\frac{Var(F)+Cov(F,X_n)}{Var(F)Var(S)}=1+\frac{Cov(F,X_n)}{Var(F)}<1$$
The final step is based on the assumption made in front. The conclusion is that the relation between $F$ and $X_n$ is negative. This is the math explanation of regression. \par

%----------------------------------------------------------------------------------------
%	SECTION 4
%----------------------------------------------------------------------------------------

\subsection{Regression \& Bivariate Normal Distribution}
The bivariate normal distribution is a generalization of the normal distribution to two dimensions. It has two random variables, $X$ and $Y$, which are normally distributed and have a correlation coefficient $\rho$. The joint probability density function of $X$ and $Y$ is given by
$$ f(x,y)=(2\pi \sigma_1\sigma_2\sqrt{1-\rho^2})^{-1}\exp [-\frac{1}{2(1-\rho^2)}\left(\frac{(x-\mu_1)}{\sigma_1^2}-\frac{2\rho(x-\mu_1)(y-\mu_2)}{\sigma_1\sigma_2}+\frac{(y-\mu_2)}{\sigma_2^2}\right)] $$
Ususally, we use $N(\mu_1,\mu_2,\sigma_1^2,\sigma_2^2,\rho)$ to denote the bivariate normal distribution. $\mu$ and $\sigma^2$ means the mean/variation of x/y, and $\rho$ means the correlation coefficient between x and y. In regression analysis, if we assume that x,y's marginal distribution is normal, then the joint distribution of x and y is bivariate normal. Conversely, the regression model can describe the linear property of bivariate distribution. \par

\end{document}
